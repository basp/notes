\documentclass{article}
\usepackage{amsmath}
\begin{document}
\begin{flushleft}
Before we look at base 2 we'll look at what happens in base 10 first.
\begin{align}
    \frac{1}{3} &= \frac{1}{3} \cdot 10^{0}\\
                &= \frac{1 \cdot 10^1}{3} \cdot 10^{-1}\\
                &= \frac{1 \cdot 10^2}{3} \cdot 10^{-2}\\
                &= \frac{1 \cdot 10^n}{3} \cdot 10^{-n}
\end{align}
As $n$ gets larger our precision increases.
\begin{align}
    \frac{1 \cdot 10^2}{3} \cdot 10^{-2} &= 0.33\\
    \frac{1 \cdot 10^5}{3} \cdot 10^{-5} &= 0.33333
\end{align}
\end{flushleft}
\end{document}
